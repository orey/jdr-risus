\documentclass[xcolor=table]{beamer}

\usepackage[utf8]{inputenc}
\usepackage[french]{babel}

% TODO 1. Remplir les champs
\newcommand{\mytitle}{Cthulhu Dark, version française}
\newcommand{\myauthor}{Olivier Rey}
\newcommand{\mysubject}{Jeu de rôles}
\newcommand{\mykeywords}{JDR,TTRPG,RPG,orey,cthulhu,lovecraft,horreur}
\newcommand{\myversion}{2.0}
\newcommand{\mycolor}{violet}     % green pour le fond noir
\newcommand{\mylinkcolor}{blue}   % red pour le fond noir
\newcommand{\myrepo}{jdr-risus}
\newcommand{\myheader}{{\Huge R}{\huge ISUS}}
\newcommand{\myreference}{OReyJdr13}

%\usepackage[orientation=portrait,size=a4,scale=1.4,debug]{beamerposter}
\usepackage[orientation=portrait,size=a4, scale=1.48]{beamerposter}

% TODO 2. Couleur des tables
\definecolor{lightgray}{gray}{0.85}     % white version; black version 0.15
\definecolor{verylightgray}{gray}{0.95} % white version; black version 0.05
\let\oldtabular\tabular
\let\endoldtabular\endtabular
\renewenvironment{tabular}{\small\rowcolors{1}{lightgray}{verylightgray}\oldtabular}{\endoldtabular}

%%\usepackage{graphicx}
\graphicspath{{../yed/}{../images/}}

% to use \textcopyleft
\usepackage{textcomp}

\usepackage{comment}

\usepackage{wrapfig}

\usepackage{amssymb}

\begin{comment}

\mode<presentation>
{
\usetheme{Berlin}
%\usetheme{Dreuw}
}
\end{comment}

% caractéristiques
\title[Risus]{Écran Risus avec les règles complètes}
\author{Olivier Rey}
\institute{The Risus Intergalactic Consortium}
\date{May 14 2022}

% paramètres beamer
% Pas de barres de navigation
\beamertemplatenavigationsymbolsempty

% Marges sur le document
\setbeamersize{text margin left=1.5cm,text margin right=1.5cm}
\setbeamertemplate{headline}{\vspace{0.5cm}}
\setbeamertemplate{footline}{\vspace{1cm}}

% PDF features
\usepackage{hyperref}
\hypersetup{
  pdftitle={\mytitle},
  pdfauthor={\myauthor},
  colorlinks=true,
  linkcolor=\mylinkcolor, % white version
  urlcolor=\mylinkcolor, % white version
  pdfsubject={\mysubject},
  pdfkeywords ={\mykeywords},
  pdfstartview={FitH},
  bookmarksopen={false},
  bookmarksnumbered={true}
}

% POur utiliser \justifying dans les columns
\usepackage{ragged2e}

% Pour les images en mode maxi
%\usepackage{background}
%\backgroundsetup{contents={}}
\usepackage{pdfpages}

%============= Mes macros

% pour mon itemize
\usepackage{enumitem}
\usepackage{pifont}
\newlist{myitemize}{enumerate}{2}
%\setlist[myitemize,1]{label=-- \arabic*:}
%\setlist[myitemize,2]{label=\ding{217} \alph*)}
%\setlist[myitemize,1]{label=--}
\setlist[myitemize,1]{label=\ding{217}}
\setlist[myitemize,2]{label=\ding{217}}

\newcommand{\mybullet}{\ding{217}}

\newcommand{\mysection}[1]{
\vspace{0.2cm}
\noindent{\color{\mycolor}\large\textbf{#1}}
{\color{\mycolor}\hrule}
\vspace{0.2cm}
}

\newcommand{\mysubsection}[1]{
\vspace{0.1cm}
\noindent{\color{\mycolor}\emph{#1}}
}

\newcommand{\deuxcolonnes}[2]{
\begin{columns}[t]\begin{column}{.5\linewidth}
\justifying
#1\end{column}\begin{column}{.5\linewidth}
\justifying
#2\end{column}\end{columns}
}

%info alignements

\begin{comment}
\begin{tabular}%
  {>{\raggedright\arraybackslash}p{3.5cm}%
   >{\centering\arraybackslash}p{3.5cm}%
   >{\raggedleft\arraybackslash}p{3.5cm}%
  }
  \lipsum[1] & \lipsum[2] & \lipsum[3]
\end{tabular}
\end{comment}


%============================================
\begin{document}


%=======================================
\begin{frame}[t]

\deuxcolonnes{%col1
\begin{center}
\includegraphics[scale=0.60]{logo-risus-fr.png}
\end{center}

%======= mysection
% TODO Création du personnage
\mysection{Création du personnage}

%--- mysubsection
\mysubsection{Les clichés}

Risus n'utilise que des dés à 6 faces. Chaque PJ a un crédit de 10D de Création (noté 10D$\Subset$) à répartir sur des Clichés (entre 3 et 10, idéalement 4) librement choisis. Classiquement, un PJ est décrit par 4 Clichés à 4D, 3D, 2D et 1D.

\vspace{0.2cm}

\begin{tabular}{lc|lcc}
\textbf{Niveau du PJ} & \textbf{Nb de dés} & \textbf{Cliché}    & \textbf{Coût Cliché} & \textbf{Gonflette ?} \\
Débutant &  1D & Normal     & 1D$\Subset$ $\rightarrow$ 1D & Simple $^{1}$ \\
Professionnel & 3D & Magique    & 2D$\Subset$ $\rightarrow$ 1D & Double $^{2}$ \\
Expert, maître & 6D & Psioniques & 2D$\Subset$ $\rightarrow$ 1D & Double $^{2}$ \\
\end{tabular}

\vspace{0.2cm}

La table ci-dessus donne une idée du niveau du personnage en fonction de son nombre de dés dans un Cliché.

Attention, tous les clichés n'ont pas tous le même coût.

%--- mysubsection
\mysubsection{Gonflette et Double Gonflette}

\vspace{0.2cm}

\begin{tabular}{p{4.15cm}p{4.15cm}}
\textbf{$^{1}$ Gonflette}    & \textbf{$^{2}$ Double Gonflette}  \\
Accord du MJ & Accord du MJ \\
Le joueur choisit le nombre de dés de Gonflette : n & Le joueur choisit le nombre de dés de Gonflette : n \\
+nD pour un round de combat & +2nD pour un round de combat \\
-nD sur le cliché à partir du round suivant &-nD sur le cliché à partir du round suivant \\
Niveau Cliché entre parenthèses (ex. : Lutteur (3)) & Niveau Cliché entre crochets (ex. : Sorcier [3]) \\
\end{tabular}

\vspace{0.2cm}

%--- mysubsection
\mysubsection{Matériel ("Tools of the Trade")}

Inclus tant que cela reste logique et raisonnable pour le Cliché.

%--- mysubsection
\mysubsection{Options de création}

\vspace{0.2cm}

\begin{tabular}{p{2cm}p{6.3cm}}
\textbf{Option}    & \textbf{Description}  \\
Coups de Bol & 1D$\Subset$ = 3 Coups de Bol ; 1 Coup de Bol donne +1D sur une action. \\
Point Faible & Un point faible du PJ validé par le MJ = +1D$\Subset$. \\
Background   & S'il est bien fait, le MJ peut donner +1D$\Subset$. \\
\end{tabular}

\vspace{0.2cm}

%======= mysection
\mysection{Les 3 mécaniques du jeu}

\begin{myitemize}
\item L'action simple : jet contre un Facteur de Difficulté (FD) ("Target Number" ou "TN" en anglais)
\item Combat : basé sur le système du duel
\item Conflits à action unique (CAU)
\end{myitemize}


%======= mysection
% TODO Action simple
\mysection{1. Action simple : jet contre un FD}
L'action est réussie si le score est supérieur ou égal au FD.

\begin{wraptable}{l}{0pt}
\begin{tabular}{cl}
\textbf{FD} & \textbf{Difficulté de l'action} \\
\textbf{5} & Facile                          \\
\textbf{10} & Défi même pour un pro           \\
\textbf{15} & Défi héroïque                   \\
\textbf{20} & Difficulté presque surhumaine   \\
\textbf{30} & Difficulté surhumaine           \\
\end{tabular}
\end{wraptable}

Important : le FD est adapté au Cliché.

Par exemple, crocheter une serrure a un FD=7 pour un Voleur (3), FD=7+5=12 pour un Agent Secret (3) et FD=7+10=17 pour un Violoniste (3) (+0/+5/+10).

%======= mysection
% TODO Combat
\mysection{2. Combat : basé sur le système du duel}

\mysubsection{Types de combats}

Généralement, l'agresseur détermine le type de combat.

\vspace{0.2cm}

\begin{tabular}{>{\centering\arraybackslash}p{2.6cm}>{\centering\arraybackslash}p{2.6cm}>{\centering\arraybackslash}p{2.6cm}}
\textbf{Type de combat} & \textbf{Type de combat} & \textbf{Type de combat}\\
Débat & Courses de chevaux & Duel aérien \\
Duel astral & Duel psychique & Duel de banjos \\
Séduction & Tribunal & Combat physique \\
Duel de danse & Jeopardy & Etc. \\
\end{tabular}

\vspace{0.2cm}

Du type de combat dépend le type de Cliché utilisé.

Note : dans un combat, il est possible d'utiliser plusieurs Clichés différents, mais le premier Cliché à 0D fait perdre le combat.

}{%col2

\mysubsection{Round de combat}

\vspace{0.2cm}

\begin{tabular}{p{3cm}p{5.1cm}}
\textbf{Processus} & \textbf{Conséquence} \\
\textbf{1. Choisir le Cliché} & Le MJ détermine si le cliché est adapté ou pas $^{3}$ $^{4}$ \\
\textbf{2. Options} & Gonflette $^{1}$, Double Gonflette $^{2}$  \\
\textbf{3. Lancer les dés} & \\
\textbf{4. Perte pour le perdant du round}  & Cliché adapté : -1D pour la suite du combat \\
                                   & Cliché inadapté : -3D pour la suite du combat (très dangereux !) \\
\textbf{5. PJ avec Cliché à 0D} & Perd le combat. Le vainqueur fait ce qu'il veut du perdant \\
\end{tabular}

\vspace{0.2cm}

Suivant le type de combat, il se peut que le défenseur n'ait pas de Cliché adapté.

\vspace{0.2cm}

\begin{tabular}{p{2.5cm}p{5.6cm}}
\textbf{$^{3}$ Cliché inadapté} & Négociation avec le MJ. Même tiré par les cheveux, le Cliché doit être utilisable. Le roleplay permet de l'utiliser \\
\textbf{$^{4}$ Aucun Cliché ne fonctionne} & 2D pour le PJ sans Cliché, +2D pour tous les autres PJs et PNJs (validation du MJ)  \\ 
\end{tabular}

\vspace{0.2cm}

\mysubsection{Récupération}

Les PJ récupèrent les dés avec de la guérison (contextuelle au type de duel).

%======= mysection
% TODO Conflits à action unique
\mysection{3. Conflits à Action Unique (CAU)}

Le CAU est souvent justifié pour une action très rapide. Il fonctionne comme un combat normal mais avec un seul jet de dés.

%======= mysection
% TODO Groupes
\mysection{Groupes}

\mysubsection{Groupe de PNJ}

Le groupe de PNJ se comporte comme un PNJ mais avec plus de dés.

\mysubsection{Groupe de PJ}

\vspace{0.2cm}

\begin{tabular}{>{\raggedright\arraybackslash}p{2.5cm}p{5.6cm}}
\textbf{Processus} & \textbf{Conséquence} \\
\textbf{1. Choisir le Chef De Groupe (CDG)} & Le CDG est celui dont un Cliché s'applique et qui a le plus de dés \\
                                   & En cas d'égalité, les joueurs désignent leur chef \\
\textbf{2. Déterminés les Clichés adaptés}  & Généralement, celui du chef de groupe est adapté, et le reste est un mélange de Clichés adaptés et inadaptés (avec accord du MJ) \\
\textbf{3. Faire le jet} & Le Cliché du CDG compte \\
   & Les Clichés adaptés/inadaptés des autres joueurs comptes uniquement s'ils font des 6 \\
\textbf{4. Round de combat perdu} & Un des membres doit se porter volontaire pour prendre les dommages (2D si au moins un des Clichés est adapté, 6D sinon) \\
   & Si un PJ s'est désigné, le groupe a droit à un bonus de vengeance $^{5}$. Dans le cas contraire, le CDG désigne celui qui prend les dommages (pas de bonus) \\
$^{5}$ Bonus de vengeance & Le Groupe a le droit de lancer deux fois plus de dés lors du round suivant pour venger le membre du Groupe qui a pris les dommages \\
Membre du groupe à 0D pendant le combat &  On attend en général la fin du combat pour se préoccuper de son sort (savoir si le groupe est vainqueur ou pas) \\
\end{tabular}

\vspace{0.2cm}

\mysubsection{Débandade}

\vspace{0.2cm}

\begin{tabular}{>{\raggedright\arraybackslash}p{4.05cm}p{4.05cm}}
Débandade du Groupe & Tous les membres du Groupe perdent 1D pour le prochain round \\
Un membre quitte le Groupe & Il se retrouve à 0D sur son Cliché et est à la merci du vainqueur \\
Le CDG quitte le Groupe & Tous les membres du Groupe perdent 1D pour le prochain round \\
Un autre Groupe se reforme alors que le CDP du Groupe précédent a pris des dommages et est passé à OD & Le nouveau groupe a droit à un bonus de vengeance $^{5}$ \\
\end{tabular}

%======= mysection
% TODO Expérience
\mysection{Expérience}

A la fin du jeu, jet de Cliché sur les Clichés utilisés durant le jeu: si tous les dés sont pairs, +1D au cliché (6D max par Cliché).

Avec accord du MJ, il est possible d'utiliser le dé gagné pour créer un nouveau Cliché à 1D. Cela peut même être fait en cas d'action exceptionnelle durant le jeu.

}
\end{frame}

%=============================== Page 2 : tableau de caractéristiques

\begin{frame}[b]

\includegraphics[page=1]{Risus-GM-Screen-fr-OReyJdr09.pdf}

\vfill

\begin{center}
\begin{tabular}{ll}
Risus en anglais  & \href{https://www.drivethrurpg.com/product/170294/Risus-The-Anything-RPG}{Risus the RPG} (c) Big Dice Games \\
Risus en français & \href{https://rouboudou.itch.io/risus}{Risus, traduction de Tristan Lhomme, plus aides de jeu} \\
Sites américains  & \href{https://www.risusrpg.com}{risusrpg.com} \\
                  & \href{https://www.risusiverse.com/}{risuiverse.com} \\
Copyleft    & \textcopyleft\ Olivier Rey 2022 \\
Référence   & \myreference\ -- Version \myversion \\
Publié sur  & \href{https://rouboudou.itch.io}{rouboudou.itch.io} \\
            & \href{https://github.com/orey/\myrepo}{github.com/orey/\myrepo} \\      
\end{tabular}
\end{center}

\end{frame}

%=============================== Page 2 : tableau de caractéristiques
\begin{frame}[b]

\mysection{Les douze travaux}

Au final, dans chaque aventure Risus, on peut compter douze types de défis (travaux).

\vspace{0.2cm}

\begin{tabular}{p{3cm}p{5.1cm} | p{3cm}p{5.1cm}}
\textbf{Type} & \textbf{Description} & \textbf{Type} & \textbf{Description} \\
1. Physique & Courir, sauter, grimper, nager, pousser, soulever & 7. Médecine & Soigner \\
2. Persuasion & Mentir, séduire, inspirer, passionner, calmer, diriger & 8. Survie & Survivre, pister, utiliser les plantes, dompter les animaux \\
3. Communication et protocole & Langages, jargons, étiquette, coutumes particulières & 9. Érudition & Savoir, suivre des conversations complexes \\
4. Détection & Remarquer, reconnaître et comprendre les indices & 10. Intrusion & Espionner, voler, pirater, se cacher, s'échapper discrètement \\
5. Conduite, chevauchée et pilotage & Poursuivre, éviter, gérer des situations difficiles & 11. Combat & Combattre, faire des exploser des trucs \\
6. Technologie & Réparer, améliorer, désactiver, utiliser & 12. Magie & Utiliser des pouvoirs surnaturels par amusement ou pour un profit \\
\end{tabular}

\vspace{0.2cm}

Au travers de leurs aventures, les PJs seront amenés à accomplir des actions dans ces différents domaines. Note : certains Clichés couvrent une bonne partie de ces domaines.

\mysection{Le dernier Cliché}

Lancez un D100.

\vspace{0.2cm}

\begin{tabular}{p{5.55cm}p{5.55cm}p{5.55cm}}
01. Alchimiste & 34. Bourreau & 67. Star du porno \\
02. Général de bureau & 35. Ex-mercenaire & 68. Prêtre \\
03. Barman & 36. Danseur de danse exotique & 69. Prophète \\
04. Star de basket & 37. Explorateur & 70. Rôdeur \\
05. Vétéran effrayé des combats & 38. Investigateur ésotérique & 71. Médium \\
06. Berserker & 39. Membre d'une fraternité & 72. Psychanalyste \\
07. Revendeur de contrebande de marché noir & 40. Ami des animaux & 73. Aide de ranch \\
08. Forgeron & 41. Géo-Trouve-Tout & 74. Rebelle \\
09. Chirurgien du cerveau & 42. Joueur & 75. Mégalomane sadique \\
10. Boucher & 43.Gigolo/Call girl & 76. Marin \\
11. Cambrioleur & 44. Groupie & 77. Saint \\
12. Chef cuisinier & 45. Plouc & 78. Vendeur \\
13. Élu d'une ancienne prophétie & 46. Hédoniste & 79. Savant \\
14. Acrobate de cirque & 47. Victime impuissante & 80. Fan de science-fiction \\
15. Ingénieur & 48. Tueur à gage & 81. Artiste sensible \\
16. Comédien/bouffon & 49. Clochard & 82. Serial killer \\
17. Geek & 50. Homme au foyer & 83. Gros crado \\
18. Arnaqueur & 51. Lâche invétéré & 84. Monsieur Je-sais-tout \\
19. Parent concerné & 52. Homme à femmes & 85. Détective sur son temps libre \\
20. Contorsionniste & 53. A abandonné ses études de droit & 86. Dilettante pourri gâté \\
21. Glandeur de canapé & 54. Pillard & 87. Magicien de scène \\
22. Icône de la scène alternative & 55. Scientifique fou & 88. Conteur \\
23. Cowboy & 56. Avare & 89. Cascadeur doublure \\
24. Informateur construisant une nouvelle vie & 57. Monstre incompris & 90. Aventurier de film de capes et d'épées \\
25. Reporter débutant & 58. Musicien & 91. Personne branchée \\
26. Cyborg machine à tuer & 59. Maniaque du rangement & 92. Lauréat de Qui Veut Gagner Des Millions \\
27. Féru de magie noire & 60. Écrivain de romans & 93. Agent secret infiltré \\
28. Danseur & 61. Homme d'extérieur & 94. Acteur sans emploi \\
29. Dictateur destitué & 62. Voyeur & 95. Vampire \\
30. Disc Jockey & 63. Philosophe & 96. Vétérinaire \\
31. Dealer de drogues & 64. Pilote & 97. Visiteur du futur \\
32. Gladiateur esclave en fuite & 65. Poète & 98. Travailleur social volontaire \\
33. Ex-taulard & 66. Activiste politique & 99. Combattant du dimanche \\
& & 00. Loup-garou \\
\end{tabular}

\mysection{Anatomie d'un Cliché}

Les Clichés ont souvent les dimensions suivantes.

\vspace{0.2cm}

\begin{tabular}{p{4.5cm}p{12.5cm}}
\textbf{Dimension} & \textbf{Exemples} \\
Profession & Détective privé (3), Pilote de chasse (3), Agent d'accueil vampire de supermarché(3), Avocat vampire (3) \\
Race ou espèce & Astronaute minotaure (4), Derviche nain (3), Elfe archer (3), Voleur de super butins (3) \\
Background culturel & Barbare (4), Sorcier langouste de Globédria (2), Astronaute Minotaure du Pays des Éclairs Violets (4) \\
Histoire personnelle & Ancien criminel (3), Patron de night-club cynique et égoïste (2), Ancien mercenaire héroïque connu pour défendre la veuve et l'orphelin (4), Français (3), Artiste connu auparavant comme prince (1) \\
Degré de dévotion & Combattant du dimanche (2), Féru de magie noire (1), Chirurgien du cerveau amateur (3), Prêtre dévoué du Dieu Chevelu (3), Médecin de combat zélé qui manque toujours de gaze (2), Le plus grand fan de Joe Dassin de l'univers (6) \\
Inclinaison relieuse ou philosophique & Astronaute minotaure breton (4), Astronaute minotaure breton hindouiste (4) \\
Classe sociale et moyens financiers & Détective privé (4), Détective privé à la rue (4), Artiste clochard (3) \\
Genre & Généralement les Clichés sont masculins mais n'hésitez pas à les féminiser \\
Affiliation à un groupe & Astronaute minotaure breton hindouiste et franc-maçon (4), Justicier sombre, chef des scouts (5) \\
Comportement & Grand-père tueur à gage enjoué (4), Tueur à gage froid et nerveux (4) \\
Apparence & Fouineur à l'œil perçant (4), Héro de football américain à la machoire carrée (3) \\
Références maladroites & Alain Delon (3), Jackie Chan (5), Sigourney Weaver (4), Sherlock (2), Romeo (4) \\
Buts & Généticien déterminé à soigner le cancer (4), Généticien voulant dominer le monde (4) \\
Image de soi & Bretteur légendaire dans ses propres rêves (2), Gentil géant convaincu qu'il ne devrait pas vivre (3) \\ 
Intrigues et relations & PIlot de chasse que l'on consulte pour des conseils amoureux (3), Archéologue globe-trotter secrètement amoureux de la rousse qui n'arrête pas de lui voler ses trouvailles (4) \\
Problèmes & Aventurier aveugle et lubrique (3), Nécromancien avec un sérieux problème au lit (4), Jeune femme anglaise hautaine (3)
\end{tabular}

\end{frame}




\end{document}


